\documentclass{article}

\usepackage{mathtools}
\usepackage{geometry}
\usepackage{graphicx}
\usepackage{titlesec}
\usepackage{listings}
\usepackage{placeins}
\usepackage{subcaption}
\usepackage[hidelinks]{hyperref}
\usepackage{cleveref}

\geometry{
  top=2cm,
  bottom=2cm,
  left=3.5cm,
  right=3.5cm
}

\titleformat{\section}
  {\large\bfseries}
  {\llap{\hspace*{-2.5em}Q\thesection:\hfill}}
  {0pt}
  {}
\titlespacing*{\section}
  {0pt}
  {2ex}
  {1ex}
\titleformat{\subsection}
  {\itshape\bfseries}
  {\llap{\hspace{-2.5em}
  \rm\bf(\alph{subsection})
  \hfill}}
  {0pt}
  {}
\titlespacing*{\subsection}
  {0pt}
  {2ex}
  {1ex}
\titleclass{\appx}{straight}[\section]
\newcounter{appx}
\renewcommand{\theappx}{\Alph{appx}}
\titleformat{\appx}
  {\large\bfseries}
  {\llap{\hspace*{-6.6em}Appendix \theappx:\hfill}}
  {0pt}
  {}
\titlespacing*{\appx}
  {0pt}
  {2ex}
  {1ex}


\newcommand{\mail}[1]{
  \href{mailto:#1}{#1}
}

\lstset{
  frame=single,
  numbers=left,
  breaklines=true,
  basicstyle=\small
}

\newcommand{\codesnippet}[5]{
  \begin{figure}[!ht]
    \lstinputlisting[
      language=#2,
      firstline=#3,
      lastline=#4,
      firstnumber=#3
    ]{#1}
  #5
  \end{figure}
}

\newcommand{\mandelbrotsource}{../mandelbrot/mandelbrot.c}
\newcommand{\plotsource}{../mandelbrot/plot.py}

\begin{document}

\begin{center}
  \textbf{
    \LARGE Homework 1 \\
    \large SF2568 - Parallel Computation for Large Scale Problems \\
    Lukas Bjarre - \mail{lbjarre@kth.se} \\
    \rule{0.75\textwidth}{0.4pt}
  }
\end{center}

\section{Difference between a \emph{process} and a \emph{processor}}
A \emph{process} is defined by the instructions that are needed to compute a problem.
A \emph{processor} refers to the physical hardware which executes a process.

\section{Performance calculation}
$10\%$ of the time the computer is operating at single processor performance, i.e. $2$ Gflops.
The remaining $90\%$ of the time all 100 processors are computing in parallel
at a performance of $200$ Gflops.
The total performance will therefore be
\begin{equation}
  0.1 \cdot 2 + 0.9 \cdot 200 = 180.2 \text{ Gflops}
\end{equation}

\section{Is a system efficiency larger than 100\% possible?}
The system efficiency $\eta_P$ for $P$ processors is given by
\begin{equation}
  \label{eq:sysefficiency}
  \eta_P = \frac{S_P}{P} = \frac{T^*_s}{PT_P}
\end{equation}
where $T^*_s$ is the fastest serial runtime
and $T_P$ is the parallel runtime for $P$ processes.
Assuming (unrealistically) no communication in the parallel implementation,
the optimal parallel runtime will be $T^*_P = T^*_sP^{-1}$.
This implementation inserted into \eqref{eq:sysefficiency} would result in
a system efficiency of $100\%$.
However, this is a lower bound for the parallel runtime,
hence $\eta_P \leq 1$.

\section{Modified Parallel Rank Sort}
Since we have 10 times as many elements as processors 10 elements can be allocated to each processor.
Each processor has to then loop through each of the allocated elements and execute the original
Parallel Rank Sort for each of them.
The resulting rank arrays then needs to be shared to the rest of the processors,
which can be done for example in a recursive doubling procedure.

\section{Parallel speedup}
\subsection{What fraction of the serial runtime is spent in serial procedures?}
We have $T^*_s = 64$ s and $T_P = 22$ s for $P=8$.
The serial fraction of the program depends on which law you apply to the numbers.
The serial fraction as given by Amdahl's law is
\begin{equation}
  \begin{aligned}
    f
    &= \frac{PT_P - T^*_s}{T^*_s\left( P - 1 \right)} \\
    &= \frac{8\cdot 22 - 64}{64\cdot 7} \\
    &= \frac{1}{4} = 0.25
  \end{aligned}
\end{equation}
The serial fraction as given by Gustafson's law is
\begin{equation}
  \begin{aligned}
    f'
    &= \frac{PT_P - T^*_s}{T_P\left( P - 1 \right)} \\
    &= \frac{8\cdot 22 - 64}{22\cdot 7} \\
    &= \frac{8}{11} \approx 0.72
  \end{aligned}
\end{equation}

\subsection{Determine the parallel speedup}
The speedup once again depends on which law you apply.
The speedup as given by Amdahl is
\begin{equation}
  \begin{aligned}
    S_P
    &= \frac{1}{\left( 1-f \right)P^{-1} + f} \\
    &= \frac{1}{0.75P^{-1} + 0.25}
  \end{aligned}
\end{equation}
For $P=8$ we have a speedup of $S_8 \approx 2.9$.
We also see that even if we let $P\rightarrow\infty$,
the speedup would only converge to $S_\infty = 4$.

The scaled speedup as given by Gustafson is
\begin{equation}
  \begin{aligned}
    S'_P
    &= f' + (1 - f')P \\
    &= \frac{8 + 3P}{11}
  \end{aligned}
\end{equation}
For $P=8$ we have a scaled speedup of $S'_8 \approx 2.9$,
i.e. the same as predicted by Amdahl.
However, unlike $S_P$, $S'_P$ grows unbounded as $P\rightarrow\infty$.

\subsection{Parallel speedup on three phased program}
We notate the serial runtimes for each program phase as $T_{pi}$,
where $i=1,\,2,\,3$ describes the different phases.
We also know the relation between each phase's serial runtime and the optimal serial runtime
as $T_{p1} = 0.2T^*_s$, $T_{p2} = 0.2T^*_s$, and $T_{p3} = 0.6T^*_s$.
In the parallel version of the program each program phase will be sped up by a optimal
factor of $k^*_{pi}$ for each program phase.
We will assume perfect speedup for each phase given the given optimal process count,
i.e. $k^*_{pi} = P^{*-1}_{pi}$ for optimal process count $P^{*-1}_{pi}$.
The total speedup will be
\begin{equation}
  \begin{aligned}
    S_P
    &= \frac{T^*_s}{T^*_s\left( 0.2k^*_{p1} + 0.2k^*_{p2} + 0.6k^*_{p3} \right)} \\
    &= \frac{1}{\frac{1}{25} + \frac{1}{50} + \frac{3}{75}} \\
    &= \frac{750}{75} \\
    &= 10
  \end{aligned}
\end{equation}

\section{Parallel Mandelbrot implementation}

\subsection*{Implementation}
My parallel implementation of the mandelbrot set calculation is similair to the one
presented in the lecture slides,
except with some small alterations to increase usability and performance.
The code can be found in the appendix.

The first change is how the data is distributed between the processors.
The version presented in the lecture slides separated the complex plane grid into vertical slices,
i.e. common $x$-coordinates and iterating over the $y$-coordinates.
However, when the data is saved we want to save rows and not columns,
i.e. common $y$-coordinates and iterating over the $x$-coordinates.
I therefore changed the data distribution to horizontal strips of the complex plane grid instead
to better accomodate the saved data structure.

The second change has also to do with the saved data structure.
I wanted to make the saved file as similair to the complex plane as possible.
Since the complex plane usually is plotted with positive complex values on the top
and the negatives on the bottom,
the first row of the saved file should contain the highest complex values
and the last row the file the lowest.
Therefore, the sign of the starting $y$-position
and the direction in which the complex values was iterated in was flipped.

Finally, the viewport in which the complex plane is evaluated was generalized.
The method used in the lecture slide assumed a viewport of $[-b,\,b] \times [-b,\,b]$,
i.e. a square with sides $2b$ placed with its center at the origin.
This was changed to a viewport of $[x_s,\,x_s+x_w] \times [y_s,\,y_s-y_h]$,
i.e. a more general rectangle placed with its upper left corner at $(x_s,\,y_s)$
and with width $x_w$ and height $y_h$.

\subsection*{Output Images}
Below are some images produced by the code.
\begin{figure}[!ht]
  \centering
  \includegraphics[width=0.45\textwidth]{../mandelbrot/full.pdf}
  \caption{Plot of the Mandelbrot set within $[-2,\,2] \times i[-2,\,2]$.}
  \label{fig:standardview}
\end{figure}

\begin{figure}[!ht]
  \centering

  \begin{subfigure}[t]{0.45\textwidth}
    \centering
    \includegraphics[width=0.9\textwidth]{../mandelbrot/zoom1.pdf}
    \caption{$[-0.970,\,-0.995] \times i[0.245,\,0.270]$}
  \end{subfigure}
  ~
  \begin{subfigure}[t]{0.45\textwidth}
    \centering
    \includegraphics[width=0.9\textwidth]{../mandelbrot/zoom2.pdf}
    \caption{$[-1.50,\,-1.55] \times i[-0.025,\,0.025]$}
  \end{subfigure}
  ~
  \begin{subfigure}[t]{0.45\textwidth}
    \centering
    \includegraphics[width=0.9\textwidth]{../mandelbrot/zoom3.pdf}
    \caption{$[0.3981,\,0.3982] \times i[0.23312,\,0.23322]$}
  \end{subfigure}
  ~
  \begin{subfigure}[t]{0.45\textwidth}
    \centering
    \includegraphics[width=0.9\textwidth]{../mandelbrot/zoom4.pdf}
    \caption{$[-0.1555,\,-0.1535] \times i[0.6522,\,0.6542]$}
  \end{subfigure}

  \caption{Various zoomed in plots on the Mandelbrot set.}
\end{figure}

\newpage
\appx{Parallel Mandelbrot Sourcecode}
Below is the source code for the program which implements the parallel version
of the Mandelbrot set calculation.
The code is quite lengthy, so it is broken down into smaller snippets containing separate functions
together with a small annotation describing its function.

\codesnippet{\mandelbrotsource}{C}{1}{8}{
Library imports and our constants.
The constants define how the color picking is set by the maximum iteration count
(the number of possible colors)
and the squared escape distance that for the semi-stable convergence definition.
Constants defining the pixel size of the viewport are also set.
}

\codesnippet{\mandelbrotsource}{C}{10}{27}{
The function which assigns colors to each pixel.
This implementation uses slightly more memory than the straight-forward approach,
but saves a couple of operations for each loop.
}

\codesnippet{\mandelbrotsource}{C}{29}{45}{
Next we have the main calculation loop for each process.
It iterates over the pixels assigned to the processor,
converts each pixel to a complex number,
and saves the calculated color value in the array $M$.
}

\codesnippet{\mandelbrotsource}{C}{47}{56}{
The function for writing the results to a file is fairly basic.
}

\codesnippet{\mandelbrotsource}{C}{58}{71}{
A small function for setting up all the MPI stuff outside of the main procedure.
This function also checks to make sure that the number of processes
evenly divide the pixel grid height.
If not, it returns a negative flag to main.
}

\codesnippet{\mandelbrotsource}{C}{73}{89}{
This function sets the variables defining the viewport,
either through argv or as defaults if no arguments are passed.
Returns a negative flag if argv is of incorrect length.
}

\codesnippet{\mandelbrotsource}{C}{91}{130}{
The main procedure calls the initializing functions,
sets up and calculates the mandelbrot for the process height,
and sends everything to the 0 rank process.
The 0 rank process saves some memory by overwriting the already allocated $M$-vector
after writing its contents to the output file
}

\FloatBarrier
\newpage
\appx{Plotting Sourcecode}
Below is the sourcecode for the Python script
used to read the data produced by the C code and produce images.
\codesnippet{\plotsource}{Python}{1}{34}{}

\end{document}

